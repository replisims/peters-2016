\PassOptionsToPackage{unicode=true}{hyperref} % options for packages loaded elsewhere
\PassOptionsToPackage{hyphens}{url}
%
\documentclass[english,doc,floatsintext]{apa6}
\usepackage{lmodern}
\usepackage{amssymb,amsmath}
\usepackage{ifxetex,ifluatex}
\usepackage{fixltx2e} % provides \textsubscript
\ifnum 0\ifxetex 1\fi\ifluatex 1\fi=0 % if pdftex
  \usepackage[T1]{fontenc}
  \usepackage[utf8]{inputenc}
  \usepackage{textcomp} % provides euro and other symbols
\else % if luatex or xelatex
  \usepackage{unicode-math}
  \defaultfontfeatures{Ligatures=TeX,Scale=MatchLowercase}
\fi
% use upquote if available, for straight quotes in verbatim environments
\IfFileExists{upquote.sty}{\usepackage{upquote}}{}
% use microtype if available
\IfFileExists{microtype.sty}{%
\usepackage[]{microtype}
\UseMicrotypeSet[protrusion]{basicmath} % disable protrusion for tt fonts
}{}
\IfFileExists{parskip.sty}{%
\usepackage{parskip}
}{% else
\setlength{\parindent}{0pt}
\setlength{\parskip}{6pt plus 2pt minus 1pt}
}
\usepackage{hyperref}
\hypersetup{
            pdftitle={Replication Report - Peters et al.~2006},
            pdfauthor={Anna Lohmann1 \& Rolf H. H. Groenwold1,2},
            pdfkeywords={keywords},
            pdfborder={0 0 0},
            breaklinks=true}
\urlstyle{same}  % don't use monospace font for urls
\usepackage{longtable,booktabs}
% Fix footnotes in tables (requires footnote package)
\IfFileExists{footnote.sty}{\usepackage{footnote}\makesavenoteenv{longtable}}{}
\usepackage{graphicx,grffile}
\makeatletter
\def\maxwidth{\ifdim\Gin@nat@width>\linewidth\linewidth\else\Gin@nat@width\fi}
\def\maxheight{\ifdim\Gin@nat@height>\textheight\textheight\else\Gin@nat@height\fi}
\makeatother
% Scale images if necessary, so that they will not overflow the page
% margins by default, and it is still possible to overwrite the defaults
% using explicit options in \includegraphics[width, height, ...]{}
\setkeys{Gin}{width=\maxwidth,height=\maxheight,keepaspectratio}
\setlength{\emergencystretch}{3em}  % prevent overfull lines
\providecommand{\tightlist}{%
  \setlength{\itemsep}{0pt}\setlength{\parskip}{0pt}}
\setcounter{secnumdepth}{0}

% set default figure placement to htbp
\makeatletter
\def\fps@figure{htbp}
\makeatother

% Manuscript styling
\usepackage{upgreek}
\captionsetup{font=singlespacing,justification=justified}

% Table formatting
\usepackage{longtable}
\usepackage{lscape}
% \usepackage[counterclockwise]{rotating}   % Landscape page setup for large tables
\usepackage{multirow}		% Table styling
\usepackage{tabularx}		% Control Column width
\usepackage[flushleft]{threeparttable}	% Allows for three part tables with a specified notes section
\usepackage{threeparttablex}            % Lets threeparttable work with longtable

% Create new environments so endfloat can handle them
% \newenvironment{ltable}
%   {\begin{landscape}\begin{center}\begin{threeparttable}}
%   {\end{threeparttable}\end{center}\end{landscape}}
\newenvironment{lltable}{\begin{landscape}\begin{center}\begin{ThreePartTable}}{\end{ThreePartTable}\end{center}\end{landscape}}

% Enables adjusting longtable caption width to table width
% Solution found at http://golatex.de/longtable-mit-caption-so-breit-wie-die-tabelle-t15767.html
\makeatletter
\newcommand\LastLTentrywidth{1em}
\newlength\longtablewidth
\setlength{\longtablewidth}{1in}
\newcommand{\getlongtablewidth}{\begingroup \ifcsname LT@\roman{LT@tables}\endcsname \global\longtablewidth=0pt \renewcommand{\LT@entry}[2]{\global\advance\longtablewidth by ##2\relax\gdef\LastLTentrywidth{##2}}\@nameuse{LT@\roman{LT@tables}} \fi \endgroup}

% \setlength{\parindent}{0.5in}
% \setlength{\parskip}{0pt plus 0pt minus 0pt}

% \usepackage{etoolbox}
\makeatletter
\patchcmd{\HyOrg@maketitle}
  {\section{\normalfont\normalsize\abstractname}}
  {\section*{\normalfont\normalsize\abstractname}}
  {}{\typeout{Failed to patch abstract.}}
\patchcmd{\HyOrg@maketitle}
  {\section{\protect\normalfont{\@title}}}
  {\section*{\protect\normalfont{\@title}}}
  {}{\typeout{Failed to patch title.}}
\makeatother
\shorttitle{Replication Report}
\keywords{keywords\newline\indent Word count: X}
\usepackage{csquotes}
\ifnum 0\ifxetex 1\fi\ifluatex 1\fi=0 % if pdftex
  \usepackage[shorthands=off,main=english]{babel}
\else
  % load polyglossia as late as possible as it *could* call bidi if RTL lang (e.g. Hebrew or Arabic)
  \usepackage{polyglossia}
  \setmainlanguage[]{english}
\fi

\title{Replication Report - Peters et al.~2006}
\author{Anna Lohmann\textsuperscript{1} \& Rolf H. H. Groenwold\textsuperscript{1,2}}
\date{}


\authornote{

Add complete departmental affiliations for each author here. Each new line herein must be indented, like this line.

Enter author note here.

The authors made the following contributions. Anna Lohmann: Conceptualization, Writing - Original Draft Preparation, Simulation implementation, Review \& Editing; Rolf H. H. Groenwold: Conceptualization - Review \& Editing.

Correspondence concerning this article should be addressed to Anna Lohmann, Postal address. E-mail: \href{mailto:a.l.lohmann@lumc.nl}{\nolinkurl{a.l.lohmann@lumc.nl}}

}

\affiliation{\vspace{0.5cm}\textsuperscript{1} Department of Clinical Epidemiology, Leiden University Medical Center, Leiden, The Netherlands\\\textsuperscript{2} Rolf's second affiliation}

\abstract{
Some summary.
}



\begin{document}
\maketitle

\hypertarget{introduction}{%
\section{Introduction}\label{introduction}}

This replication report documents the replication attempt of Peters, J. L. (2006). Comparison of Two Methods to Detect Publication Bias in Meta-analysis. JAMA, 295(6), 676. \url{https://doi.org/10.1001/jama.295.6.676}

Section 2 will detail the sources of information utilized for the present replication attempt.
Section 3 will provide an overview of the information that was extracted from those sources.
Section 4 covers all Researcher degrees of freedom i.e.~decisions that had to be made by the replicators because of insufficient or contradicting information in the original sources.
Section 5 presents descriptive statistics from the data generating mechanism, i.e.~the artificial sample.
Section 6 presents the replicated results.

\hypertarget{information-basis}{%
\section{Information basis}\label{information-basis}}

The information upon which the replication was based stems from two different sources (1) the published manuscript as well as (2) a technical report.
The published manuscript mentioned that details of the simulation were available in a technical report.
This technical report is listed in the reference section of the published article, however it was not obtainable from the public domain (i.e.~online line supplements or a public online repository).
The technical report {[}Peters, J. L., Sutton, A. J., Jones, D. R., Abrams, K. R., \& Rushton, L. (2005). Performance of tests and adjustments for publication bias in the presence of heterogeneity (Technical Report No.~05--01; pp.~1--57). Department of Health Sciences, University of Leicester.{]} was hence obtained by email from the Department of Health Sciences at Leister University (\href{mailto:hsenquiries@leicester.ac.uk}{\nolinkurl{hsenquiries@leicester.ac.uk}}). Conflicting or insufficient information from there two sources were supplemented with information from referenced articles in either manuscript as well as similar publications on the topic by the same authors.

\hypertarget{extracted-information}{%
\section{Extracted Information}\label{extracted-information}}

The following information pertaining to the implementation of the simulation study was extracted from the above mentioned sources:

\hypertarget{data-generating-mechanism}{%
\subsection{Data generating mechanism}\label{data-generating-mechanism}}

\hypertarget{fixed-effects-model-is-given-by}{%
\subsubsection{fixed effects model is given by}\label{fixed-effects-model-is-given-by}}

\(y_i=\theta + \epsilon_i\)
where \(\theta\) is the true underlying effect lnOR

\hypertarget{random-effects-model}{%
\subsubsection{random effects model}\label{random-effects-model}}

\(y_i = \theta_i + \epsilon_i\)
with \(\theta_i~ N(\mu,\tau^2)\)
where \(\theta_i\) is the true effect in study \(i\)
\(\mu\) true underlying effect lnOR
\(\tau^2\) is the between-study variance

Between-study variance is defined to be 20\%, 150, and 500\% of the average within-study variance for studies from the corresponding simulations
this compares with specifications of \(I^2\), describing the percentage of total variation across studies that is due to between-study heterogeneity rather than chance (ref 25). Here 20\%, 150\% and 500 \% of the within-study variation corresponds to an \(I^2\) of 16.7\%, 60\% and 83\% respectively.

\hypertarget{method}{%
\subsection{Method}\label{method}}

\hypertarget{data-generating-mechanism-1}{%
\subsubsection{Data generating mechanism}\label{data-generating-mechanism-1}}

1000 repetitions per scenario

\hypertarget{simulation-factors}{%
\subsubsection{Simulation factors}\label{simulation-factors}}

The following table shows an overview of simulation factors.

\begin{longtable}[]{@{}lllll@{}}
\toprule
\begin{minipage}[b]{0.20\columnwidth}\raggedright
Simulation factor\strut
\end{minipage} & \begin{minipage}[b]{0.35\columnwidth}\raggedright
Levels\strut
\end{minipage} & \begin{minipage}[b]{0.26\columnwidth}\raggedright
Implementation details\strut
\end{minipage} & \begin{minipage}[b]{0.04\columnwidth}\raggedright
Source\strut
\end{minipage} & \begin{minipage}[b]{0.01\columnwidth}\raggedright
\strut
\end{minipage}\tabularnewline
\midrule
\endhead
\begin{minipage}[t]{0.20\columnwidth}\raggedright
Varied\strut
\end{minipage} & \begin{minipage}[t]{0.35\columnwidth}\raggedright
\strut
\end{minipage} & \begin{minipage}[t]{0.26\columnwidth}\raggedright
\strut
\end{minipage} & \begin{minipage}[t]{0.04\columnwidth}\raggedright
\strut
\end{minipage} & \begin{minipage}[t]{0.01\columnwidth}\raggedright
\strut
\end{minipage}\tabularnewline
\begin{minipage}[t]{0.20\columnwidth}\raggedright
Publication bias\strut
\end{minipage} & \begin{minipage}[t]{0.35\columnwidth}\raggedright
none, effect size based moderate (14\%), effect size based severe (40\%), p-value based moderate, p-value based severe (see table below)\strut
\end{minipage} & \begin{minipage}[t]{0.26\columnwidth}\raggedright
\strut
\end{minipage} & \begin{minipage}[t]{0.04\columnwidth}\raggedright
\strut
\end{minipage} & \begin{minipage}[t]{0.01\columnwidth}\raggedright
\strut
\end{minipage}\tabularnewline
\begin{minipage}[t]{0.20\columnwidth}\raggedright
True effect size (OR)\strut
\end{minipage} & \begin{minipage}[t]{0.35\columnwidth}\raggedright
1, 1.2, 1.5, 3, 5\strut
\end{minipage} & \begin{minipage}[t]{0.26\columnwidth}\raggedright
\strut
\end{minipage} & \begin{minipage}[t]{0.04\columnwidth}\raggedright
\strut
\end{minipage} & \begin{minipage}[t]{0.01\columnwidth}\raggedright
\strut
\end{minipage}\tabularnewline
\begin{minipage}[t]{0.20\columnwidth}\raggedright
Between Study Heterogeneity (\(I^2\))\strut
\end{minipage} & \begin{minipage}[t]{0.35\columnwidth}\raggedright
0, 20, 150, 500 (0, 16.7\%, 60\%, 83.3\%)\strut
\end{minipage} & \begin{minipage}[t]{0.26\columnwidth}\raggedright
percentage of the average within-study variance estimate\strut
\end{minipage} & \begin{minipage}[t]{0.04\columnwidth}\raggedright
\strut
\end{minipage} & \begin{minipage}[t]{0.01\columnwidth}\raggedright
\strut
\end{minipage}\tabularnewline
\begin{minipage}[t]{0.20\columnwidth}\raggedright
Number of primary studies in meta-analysis\strut
\end{minipage} & \begin{minipage}[t]{0.35\columnwidth}\raggedright
6, 16, 30, 90\strut
\end{minipage} & \begin{minipage}[t]{0.26\columnwidth}\raggedright
this number corresponds to the number of studies after publication bias\strut
\end{minipage} & \begin{minipage}[t]{0.04\columnwidth}\raggedright
\strut
\end{minipage} & \begin{minipage}[t]{0.01\columnwidth}\raggedright
\strut
\end{minipage}\tabularnewline
\begin{minipage}[t]{0.20\columnwidth}\raggedright
Sample size control group\strut
\end{minipage} & \begin{minipage}[t]{0.35\columnwidth}\raggedright
exponential of the normal distribution with a mean of 5 and variance of 0.3\strut
\end{minipage} & \begin{minipage}[t]{0.26\columnwidth}\raggedright
\strut
\end{minipage} & \begin{minipage}[t]{0.04\columnwidth}\raggedright
\strut
\end{minipage} & \begin{minipage}[t]{0.01\columnwidth}\raggedright
\strut
\end{minipage}\tabularnewline
\begin{minipage}[t]{0.20\columnwidth}\raggedright
Fixed\strut
\end{minipage} & \begin{minipage}[t]{0.35\columnwidth}\raggedright
\strut
\end{minipage} & \begin{minipage}[t]{0.26\columnwidth}\raggedright
\strut
\end{minipage} & \begin{minipage}[t]{0.04\columnwidth}\raggedright
\strut
\end{minipage} & \begin{minipage}[t]{0.01\columnwidth}\raggedright
\strut
\end{minipage}\tabularnewline
\begin{minipage}[t]{0.20\columnwidth}\raggedright
Ratio treatment:control group\strut
\end{minipage} & \begin{minipage}[t]{0.35\columnwidth}\raggedright
1:1\strut
\end{minipage} & \begin{minipage}[t]{0.26\columnwidth}\raggedright
\strut
\end{minipage} & \begin{minipage}[t]{0.04\columnwidth}\raggedright
\strut
\end{minipage} & \begin{minipage}[t]{0.01\columnwidth}\raggedright
\strut
\end{minipage}\tabularnewline
\begin{minipage}[t]{0.20\columnwidth}\raggedright
Probability of event in control group\strut
\end{minipage} & \begin{minipage}[t]{0.35\columnwidth}\raggedright
sampled from unif(0.3, 0.7)\strut
\end{minipage} & \begin{minipage}[t]{0.26\columnwidth}\raggedright
\strut
\end{minipage} & \begin{minipage}[t]{0.04\columnwidth}\raggedright
\strut
\end{minipage} & \begin{minipage}[t]{0.01\columnwidth}\raggedright
\strut
\end{minipage}\tabularnewline
\bottomrule
\end{longtable}

\hypertarget{publication-bias}{%
\paragraph{Publication bias}\label{publication-bias}}

\begin{enumerate}
\def\labelenumi{\arabic{enumi}.}
\tightlist
\item
  studies are censored as a result of the one-sided p-value associated with the effect estimate of interest
\item
  studies with the most extreme effect estimates of effect are censored
\end{enumerate}

\hypertarget{compared-methods}{%
\subsubsection{Compared Methods}\label{compared-methods}}

\hypertarget{model-1-eggers-fixed-effects-regression-on-the-standard-error}{%
\paragraph{Model 1 ( Egger's fixed effects regression on the standard error)}\label{model-1-eggers-fixed-effects-regression-on-the-standard-error}}

Egger's regression test is given by
\(\frac{y_i}{se_i}= \beta + \frac{\alpha}{se_i}+\epsilon_i\)

which is equivalent to
\(y_i = \alpha +\beta \cdot se_i +\epsilon_i \cdot se_i\) weighted by \(\frac{1}{se_i^2}\)

where \(y_i\) is the lnOR from study i and \(se_i\) is the standard error of \(y_i\)

\hypertarget{model-2-eggerts-fixed-regression-on-the-inverse-of-sample-size}{%
\paragraph{Model 2 (Eggert's fixed regression on the inverse of sample size)}\label{model-2-eggerts-fixed-regression-on-the-inverse-of-sample-size}}

\(y_i \cdot size_i = \alpha +\beta \cdot size_i +\epsilon_i\)

\hypertarget{performance-measures}{%
\subsection{Performance measures}\label{performance-measures}}

Type 1 error rate (proportion of false positives)
Power to detect publication bias when it is present (proportion of true positive results)

\hypertarget{researcher-degrees-of-freedom}{%
\section{Researcher degrees of Freedom}\label{researcher-degrees-of-freedom}}

\begin{longtable}[]{@{}lll@{}}
\toprule
\begin{minipage}[b]{0.32\columnwidth}\raggedright
Not specified\strut
\end{minipage} & \begin{minipage}[b]{0.44\columnwidth}\raggedright
Replicator decision\strut
\end{minipage} & \begin{minipage}[b]{0.16\columnwidth}\raggedright
Justification\strut
\end{minipage}\tabularnewline
\midrule
\endhead
\begin{minipage}[t]{0.32\columnwidth}\raggedright
Dealing with empty cells\strut
\end{minipage} & \begin{minipage}[t]{0.44\columnwidth}\raggedright
add 0.5 to every empty cell\strut
\end{minipage} & \begin{minipage}[t]{0.16\columnwidth}\raggedright
common solution\strut
\end{minipage}\tabularnewline
\begin{minipage}[t]{0.32\columnwidth}\raggedright
Which set to compute average within-study-variance on\strut
\end{minipage} & \begin{minipage}[t]{0.44\columnwidth}\raggedright
largest number of studies generated before application of publication bias\strut
\end{minipage} & \begin{minipage}[t]{0.16\columnwidth}\raggedright
Most accurate correspondance to intended I\^{}2\strut
\end{minipage}\tabularnewline
\begin{minipage}[t]{0.32\columnwidth}\raggedright
Data dependence\strut
\end{minipage} & \begin{minipage}[t]{0.44\columnwidth}\raggedright
each scenario is implemented in independently generated data\strut
\end{minipage} & \begin{minipage}[t]{0.16\columnwidth}\raggedright
Recomendatios of Burton et al.\strut
\end{minipage}\tabularnewline
\begin{minipage}[t]{0.32\columnwidth}\raggedright
Is probability of event in control-group fixed for all studies in one meta-analysis?\strut
\end{minipage} & \begin{minipage}[t]{0.44\columnwidth}\raggedright
Probability of event in CG assumed as fixed\strut
\end{minipage} & \begin{minipage}[t]{0.16\columnwidth}\raggedright
Wording was not unambigous but both authors tended towards that interpretation\strut
\end{minipage}\tabularnewline
\bottomrule
\end{longtable}

\hypertarget{publication-bias-based-on-effect-size}{%
\subsection{Publication bias based on effect size}\label{publication-bias-based-on-effect-size}}

two levels
moderate vs severe
\textgreater{} either 14\% or 40 \% of the most extreme studies showing a negative effect of the exposure (i.e.~OR \textless{}1)
were censored such that the final number of studies in a meta -analysis was still 6, 16, 30, or 90
i.e.~for the 6 studies 10 haven been generated and 4 studies with the most extreme negative estimates have been censored

This statement contradicts itself. On the one hand it suggests, that extreme studies with a negative effect of the exposure should be censored.
On the other hand it suggests to censor either 14\% or 40\% of studies. Especially with large effect sized (e.g.~an OR of 5) it is highly unlikely to have 40\% of studies with a negative effect of the exposure.

\hypertarget{simulation-descriptives}{%
\section{Simulation Descriptives}\label{simulation-descriptives}}

\hypertarget{results}{%
\section{Results}\label{results}}

\hypertarget{discussion}{%
\section{Discussion}\label{discussion}}

\hypertarget{acknowledgments}{%
\section{Acknowledgments}\label{acknowledgments}}

\newpage

\newpage

\hypertarget{references}{%
\section{References}\label{references}}

\begingroup
\setlength{\parindent}{-0.5in}
\setlength{\leftskip}{0.5in}

\hypertarget{refs}{}

\newpage

\hypertarget{reproducibility-information}{%
\subsubsection{Reproducibility Information}\label{reproducibility-information}}

This report was last updated on 2020-08-04 20:35:51.
The simulation replication was conducted using the following computational environment and dependencies:

\begin{verbatim}
#> - Session info ---------------------------------------------------------------
#>  setting  value                       
#>  version  R version 3.6.2 (2019-12-12)
#>  os       Ubuntu 18.04.3 LTS          
#>  system   x86_64, linux-gnu           
#>  ui       X11                         
#>  language (EN)                        
#>  collate  en_US.UTF-8                 
#>  ctype    en_US.UTF-8                 
#>  tz       Europe/Berlin               
#>  date     2020-08-04                  
#> 
#> - Packages -------------------------------------------------------------------
#>  package     * version    date       lib source                      
#>  assertthat    0.2.1      2019-03-21 [1] CRAN (R 3.6.2)              
#>  backports     1.1.8      2020-06-17 [1] CRAN (R 3.6.2)              
#>  bookdown      0.20       2020-06-23 [1] CRAN (R 3.6.2)              
#>  callr         3.4.3      2020-03-28 [1] CRAN (R 3.6.2)              
#>  cli           2.0.2      2020-02-28 [1] CRAN (R 3.6.2)              
#>  crayon        1.3.4      2017-09-16 [1] CRAN (R 3.6.2)              
#>  desc          1.2.0      2018-05-01 [1] CRAN (R 3.6.2)              
#>  devtools      2.3.0      2020-04-10 [1] CRAN (R 3.6.2)              
#>  digest        0.6.25     2020-02-23 [1] CRAN (R 3.6.2)              
#>  ellipsis      0.3.1      2020-05-15 [1] CRAN (R 3.6.2)              
#>  evaluate      0.14       2019-05-28 [1] CRAN (R 3.6.2)              
#>  fansi         0.4.1      2020-01-08 [1] CRAN (R 3.6.2)              
#>  fs            1.4.1      2020-04-04 [1] CRAN (R 3.6.2)              
#>  glue          1.4.1      2020-05-13 [1] CRAN (R 3.6.2)              
#>  htmltools     0.5.0      2020-06-16 [1] CRAN (R 3.6.2)              
#>  knitr         1.29       2020-06-23 [1] CRAN (R 3.6.2)              
#>  magrittr      1.5        2014-11-22 [1] CRAN (R 3.6.2)              
#>  memoise       1.1.0      2017-04-21 [1] CRAN (R 3.6.2)              
#>  papaja      * 0.1.0.9997 2020-08-04 [1] Github (crsh/papaja@0457653)
#>  pkgbuild      1.0.8      2020-05-07 [1] CRAN (R 3.6.2)              
#>  pkgload       1.0.2      2018-10-29 [1] CRAN (R 3.6.2)              
#>  prettyunits   1.1.1      2020-01-24 [1] CRAN (R 3.6.2)              
#>  processx      3.4.2      2020-02-09 [1] CRAN (R 3.6.2)              
#>  ps            1.3.3      2020-05-08 [1] CRAN (R 3.6.2)              
#>  R6            2.4.1      2019-11-12 [1] CRAN (R 3.6.2)              
#>  remotes       2.1.1      2020-02-15 [1] CRAN (R 3.6.2)              
#>  rlang         0.4.7      2020-07-09 [1] CRAN (R 3.6.2)              
#>  rmarkdown     2.3        2020-06-18 [1] CRAN (R 3.6.2)              
#>  rprojroot     1.3-2      2018-01-03 [1] CRAN (R 3.6.2)              
#>  sessioninfo   1.1.1      2018-11-05 [1] CRAN (R 3.6.2)              
#>  stringi       1.4.6      2020-02-17 [1] CRAN (R 3.6.2)              
#>  stringr       1.4.0      2019-02-10 [1] CRAN (R 3.6.2)              
#>  testthat      2.3.2      2020-03-02 [1] CRAN (R 3.6.2)              
#>  usethis       1.6.1      2020-04-29 [1] CRAN (R 3.6.2)              
#>  withr         2.2.0      2020-04-20 [1] CRAN (R 3.6.2)              
#>  xfun          0.16       2020-07-24 [1] CRAN (R 3.6.2)              
#>  yaml          2.2.1      2020-02-01 [1] CRAN (R 3.6.2)              
#> 
#> [1] /home/anna/R/x86_64-pc-linux-gnu-library/3.6
#> [2] /usr/local/lib/R/site-library
#> [3] /usr/lib/R/site-library
#> [4] /usr/lib/R/library
\end{verbatim}

The current Git commit details are:

\begin{verbatim}
#> Local:    master /home/anna/Dropbox/anna/projects/replisims/peters2006
#> Remote:   master @ origin (https://github.com/replisims/peters-2016.git)
#> Head:     [7af113c] 2020-08-02: Update replication report
\end{verbatim}


\end{document}
